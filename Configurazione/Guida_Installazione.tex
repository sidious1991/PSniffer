%%%%%%%%%%%%%%%%%%%%%%%%%%%%%%%%%%%%%%%%%
% Simple Sectioned Essay Template
% LaTeX Template
%
% This template has been downloaded from:
% http://www.latextemplates.com
%
% Note:
% The \lipsum[#] commands throughout this template generate dummy text
% to fill the template out. These commands should all be removed when 
% writing essay content.
%
%%%%%%%%%%%%%%%%%%%%%%%%%%%%%%%%%%%%%%%%%

%----------------------------------------------------------------------------------------
%	PACKAGES AND OTHER DOCUMENT CONFIGURATIONS
%----------------------------------------------------------------------------------------

\documentclass[12pt]{article} % Default font size is 12pt, it can be changed here

\usepackage{geometry} % Required to change the page size to A4
\geometry{a4paper} % Set the page size to be A4 as opposed to the default US Letter

\usepackage{graphicx} % Required for including pictures

\usepackage{float} % Allows putting an [H] in \begin{figure} to specify the exact location of the figure
\usepackage{wrapfig} % Allows in-line images such as the example fish picture

\usepackage{lipsum} % Used for inserting dummy 'Lorem ipsum' text into the template
\usepackage{subfigure}

\usepackage{url}
\usepackage{hyperref}

\linespread{1.2} % Line spacing

%\setlength\parindent{0pt} % Uncomment to remove all indentation from paragraphs

\graphicspath{{Pictures/}} % Specifies the directory where pictures are stored

\begin{document}

%----------------------------------------------------------------------------------------
%	TITLE PAGE
%----------------------------------------------------------------------------------------

\begin{titlepage}

\newcommand{\HRule}{\rule{\linewidth}{0.5mm}} % Defines a new command for the horizontal lines, change thickness here

\center % Center everything on the page

\textsc{\LARGE Universit\'a Tor Vergata}\\[1.5cm] % Name of your university/college
\textsc{\Large Corso di Laurea di Ingegneria Informatica}\\[0.5cm] % Major heading such as course name
\textsc{\large A.A. 2015-2016}\\[0.5cm] % Minor heading such as course title

\HRule \\[0.4cm]
{ \huge \bfseries Mobile Sniffing Tool}\\[0.4cm] % Title of your document
{ \bfseries Guida all'installazione}\\[0,4cm]
\HRule \\[1.5cm]

\begin{minipage}{0.4\textwidth}
\begin{flushleft} \large
\emph{Autori:}\\
Paolo \textsc{Salom\'e}\\
Stefano \textsc{Agostini} % Your name
\end{flushleft}
\end{minipage}
~
\begin{minipage}{0.4\textwidth}
\begin{flushright} \large
\emph{Supervisori:}\\
Prof. Giuseppe \textsc{Italiano}\\ 
Dr. Marco \textsc{Querini}  % Supervisor's Name
\end{flushright}
\end{minipage}\\[8,5cm]

{\large \today}\\[3cm] % Date, change the \today to a set date if you want to be precise

%\includegraphics{Logo}\\[1cm] % Include a department/university logo - this will require the graphicx package

\vfill % Fill the rest of the page with whitespace

\end{titlepage}

\newpage % Begins the essay on a new page instead of on the same page as the table of contents 

%----------------------------------------------------------------------------------------
%	GUIDE
%----------------------------------------------------------------------------------------

L'installazione del nostro \textit{software Android} consta dei seguenti passi: 

\begin{enumerate}
\item Effettuare l'accesso ai privilegi \textit{root} del telefono.
\item Installare un qualsiasi emulatore di un terminale 
(\href{https://play.google.com/store/apps/details?id=jackpal.androidterm}{qui un possibile emulatore}).
\item Per integrare altri comandi \textit{bash} utilizzati dalla nostra applicazione installare \textit{BusyBox} 
(\href{https://play.google.com/store/apps/details?id=stericson.busybox&hl=it}{qui il link}). 
\item Inserire l'allegato eseguibile \textit{tcpdump} in una \textit{directory} a scelta. Ipotizzando che tale \textit{directory} sia \textit{/sdcard/data}, eseguire da \textit{terminale} i seguenti comandi:


\begin{itemize}

\item \textbf{su}
\item \textbf{mount -o remount,rw /system}
\item \textbf{cp /sdcard/data/tcpdump system/bin}
\item \textbf{cd system/bin}
\item \textbf{chmod 777 tcpdump}
\item \textbf{mount -o remount,ro /system}

\end{itemize}
  
\item A questo punto \'e possibile installare l'\textit{apk} fornita come allegato (navigare nella cartella del progetto in \textit{Sicurezza/app/build/outputs/apk}) oppure eseguire l'\textit{app} mediante \textit{AndroidStudio}.

\end{enumerate}

La nostra applicazione \'e stata testata su un dispositivo mobile \textit{Sony Xperia LT30p con Android 4.3}.

\end{document}
